\section{Phương pháp}

\subsection{Bộ dữ liệu: UIT-HWDB-word}

\subsubsection{Tổng quan bộ dữ liệu}
Ta sử dụng bộ dữ liệu UIT-HWDB-word, một phần của bộ dữ liệu chữ viết tay tiếng Việt UIT-HWDB (\cite{DBLP:conf/rivf/NguyenVN22}).

Một số mẫu chữ viết tay từ bộ dữ liệu được minh họa trong Hình~\ref{fig:uit_hwdb_samples}.

\begin{figure}[H]
    \centering
    \begin{minipage}{0.45\textwidth}
        \centering
        \includegraphics[width=\linewidth]{img/ex_train.png}
        \caption{(a) Tập huấn luyện}
    \end{minipage}\hfill
    \begin{minipage}{0.45\textwidth}
        \centering
        \includegraphics[width=\linewidth]{img/ex_test.png}
        \caption{(b) Tập kiểm thử}
    \end{minipage}
    \caption{Một số mẫu chữ viết tay từ bộ dữ liệu UIT-HWDB-word}
    \label{fig:uit_hwdb_samples}
\end{figure}


\subsubsection{Cấu trúc dữ liệu}
Bộ dữ liệu được tổ chức thành hai tập chính: tập huấn luyện (train) và tập kiểm tra (test). Cấu trúc thư mục như sau:
\begin{verbatim}
UIT_HWDB_word/
    train_data/
        1/
            image_001.png
            ...
            label.json
        ...
    test_data/
        250/
            ...
            label.json
        ...
\end{verbatim}

Mỗi thư mục con chứa các ảnh chữ viết tay và một file \texttt{label.json} chứa nhãn tương ứng cho từng ảnh. Định dạng của file nhãn như sau:
\begin{verbatim}
{
    "1.jpg": "xin chào",
    "2.jpg": "cảm ơn",
    ...
}
\end{verbatim}

\subsection{Pipeline}
Quy trình fine-tuning mô hình DeepSeek-OCR trên bộ dữ liệu UIT-HWDB-word bao gồm các bước chính:
\begin{enumerate}
    \item Chuẩn bị dữ liệu: Chuẩn bị bộ dữ liệu UIT-HWDB-word với ảnh và nhãn tương ứng.
    \item Tiền xử lý dữ liệu: Tiền xử lý ảnh và định dạng dữ liệu theo chuẩn hội thoại.
    \item Huấn luyện: Huấn luyện mô hình DeepSeek-OCR trên tập huấn luyện.
    \item Đánh giá: Đánh giá mô hình trên tập kiểm tra sử dụng các metrics như CER, Accuracy và Exact Match.
\end{enumerate}
Tuy nhiên, trước khi vào pipeline chính, ta sẽ tiến hành phân tích dữ liệu để hiểu rõ hơn về đặc điểm của bộ dữ liệu và từ đó đưa ra các quyết định tiền xử lý và cấu hình mô hình phù hợp.

\subsection{Phân tích dữ liệu}
Quá trình phân tích dữ liệu được thực hiện chi tiết trên tập huấn luyện (107,607 mẫu) và tập kiểm tra (2,881 mẫu) nhằm định hướng cho các quyết định tiền xử lý và cấu hình mô hình.

\subsubsection{Đặc điểm hình ảnh}
Phân tích thống kê kích thước ảnh cho thấy những đặc trưng quan trọng:
\begin{itemize}
    \item Chiều rộng cố định: Tất cả các ảnh trong bộ dữ liệu đều có chiều rộng 128 pixels.
    \item Chiều cao biến thiên: Chiều cao ảnh có sự dao động lớn, từ 23 pixels đến 974 pixels, với giá trị trung bình khoảng 98 pixels và trung vị là 93 pixels.
    \item Điểm ngoại lai (Outliers): Mặc dù phần lớn ảnh có chiều cao dưới 128 pixels, sự tồn tại của các ảnh có chiều cao lên tới gần 1000 pixels cho thấy có những mẫu chữ viết tay rất dài hoặc được viết theo chiều dọc.
\end{itemize}
Từ kết quả này, việc lựa chọn kích thước ảnh đầu vào cho mô hình là 384x384 được đánh giá là tối ưu. Kích thước này đủ lớn để chứa trọn vẹn đa số các ảnh (với chiều cao trung bình $\sim$98px) mà không cần co giãn quá nhiều, đồng thời hạn chế lãng phí tài nguyên tính toán cho phần padding của các ảnh nhỏ.

\begin{figure}[H]
    \centering
    \includegraphics[width=\linewidth]{img/img_size.png}
    \caption{Phân bố kích thước ảnh trong bộ dữ liệu UIT-HWDB-word}
    \label{fig:img_size}
\end{figure}

\subsubsection{Phân bố nhãn}
\begin{itemize}
    \item Phân phối Long-tail: Tần suất xuất hiện của các nhãn từ tuân theo quy luật phân phối đuôi dài. Một lượng nhỏ các từ thông dụng chiếm tỷ trọng lớn trong tập dữ liệu, trong khi có những từ chỉ xuất hiện rất ít lần.
\end{itemize}

\begin{figure}[H]
    \centering
    \includegraphics[width=\linewidth]{img/label_dist.png}
    \caption{Phân bố nhãn}
    \label{fig:label_dist}
\end{figure}

\subsection{Tiền xử lý dữ liệu}
Dựa trên notebook hướng dẫn fine-tuning mô hình DeepSeek-OCR của Unsloth (\cite{unsloth}), ta sẽ thực hiện các bước tiền xử lý dữ liệu, bao gồm chuẩn bị dữ liệu, định dạng hội thoại và xử lý ảnh.
\subsubsection{Chuẩn bị dữ liệu}
Quá trình chuẩn bị dữ liệu bao gồm:
\begin{enumerate}
    \item Đọc các file \texttt{label.json} từ tập dữ liệu
    \item Tải ảnh và chuyển đổi sang định dạng RGB.
\end{enumerate}

\subsubsection{Định dạng hội thoại}
Để fine-tune mô hình DeepSeek-OCR (một mô hình Vision-Language Model), dữ liệu được chuyển đổi sang định dạng hội thoại (conversation format) chuẩn:
\begin{verbatim}
{
    "messages": [
        {
            "role": "<|User|>",
            "content": "<image>\nFree OCR.",
            "images": [<PIL.Image>]
        },
        {
            "role": "<|Assistant|>",
            "content": "nhãn_của_ảnh"
        }
    ]
}
\end{verbatim}
Trong đó, token \texttt{<image>} đại diện cho vị trí chèn các embedding của ảnh, và câu lệnh "Free OCR." đóng vai trò là prompt hướng dẫn mô hình thực hiện tác vụ nhận dạng ký tự quang học.

\subsubsection{Xử lý ảnh}
Trong quá trình huấn luyện, ảnh đầu vào được xử lý thông qua \texttt{DeepSeekOCRDataCollator}. Các tham số chính được thiết lập như sau:
\begin{itemize}
    \item \texttt{crop\_mode=False}: Tắt chế độ cắt ảnh động (dynamic cropping). DeepSeek-OCR mặc định hỗ trợ cắt ảnh lớn thành nhiều mảnh để xử lý chi tiết. Tuy nhiên, với bộ dữ liệu UIT-HWDB-word gồm các ảnh từ đơn lẻ kích thước nhỏ, việc này không cần thiết. Thiết lập \texttt{False} giúp mô hình tập trung vào toàn cục (global view) và giảm chi phí tính toán.
    \item \texttt{base\_size=384}: Kích thước khung hình cơ sở. Ảnh đầu vào được thay đổi kích thước (resize) hoặc thêm viền (padding) để đạt kích thước $384 \times 384$ pixels. Giá trị 384 được chọn dựa trên phân tích phân bố kích thước ảnh, đảm bảo bao quát được phần lớn các mẫu dữ liệu (chiều cao trung bình $\sim$98px) mà không cần co nhỏ làm mất thông tin chi tiết.
    \item \texttt{image\_size=384}: Kích thước chuẩn hóa đầu vào cho Vision Encoder. Trong chế độ không cắt ảnh, tham số này đồng bộ với \texttt{base\_size} để định hình tensor đầu vào cho mạng nơ-ron.
\end{itemize}

\subsection{Cấu hình huấn luyện}

\subsubsection{Mô hình cơ sở}
Ta sử dụng mô hình DeepSeek-OCR thông qua thư viện Unsloth (\texttt{FastVisionModel}). Mô hình được huấn luyện ở độ chính xác 16-bit (BF16/FP16) với thiết lập \texttt{load\_in\_4bit = False}.

\subsubsection{LoRA}
Vì tài nguyên có hạn, ta sử dụng kỹ thuật LoRA (Low-Rank Adaptation). Ta sẽ chỉ tinh chỉnh các ma trận hạng thấp (adapters) được thêm vào các lớp attention và feed-forward, trong khi đóng băng toàn bộ trọng số của mô hình gốc. Chi tiết cấu hình được trình bày trong Bảng \ref{tab:lora_config}.
\begin{table}[H]
    \centering
    \caption{Cấu hình LoRA}
    \label{tab:lora_config}
    \begin{tabular}{|l|p{5cm}|}
        \hline
        \textbf{Tham số} & \textbf{Giá trị}                                                              \\
        \hline
        Rank (r)         & 16                                                                            \\
        \hline
        Alpha            & 16                                                                            \\
        \hline
        Dropout          & 0                                                                             \\
        \hline
        Target Modules   & \texttt{q\_proj, k\_proj, v\_proj, o\_proj, gate\_proj, up\_proj, down\_proj} \\
        \hline
    \end{tabular}
\end{table}

\subsubsection{Hyperparameter}
Quá trình fine-tuning được thực hiện với các hyperparameter được liệt kê trong Bảng \ref{tab:hyperparameters}.

\begin{table}[H]
    \centering
    \caption{Hyperparameters}
    \label{tab:hyperparameters}
    \begin{tabular}{|l|l|}
        \hline
        \textbf{Tham số}            & \textbf{Giá trị}              \\
        \hline
        Số epochs                   & 2                             \\
        \hline
        Batch size                  & 32 (per device)               \\
        \hline
        Gradient Accumulation Steps & 2                             \\
        \hline
        Learning Rate               & 1e-4                          \\
        \hline
        Optimizer                   & AdamW 8-bit                   \\
        \hline
        Scheduler                   & Linear decay (5 warmup steps) \\
        \hline
        Độ chính xác                & BF16 hoặc FP16                \\
        \hline
    \end{tabular}
\end{table}

\subsection{Môi trường huấn luyện}
Quá trình fine-tuning được tiến hành trên nền tảng \href{https://www.kaggle.com/}{Kaggle} với cấu hình phần cứng bao gồm hai GPU NVIDIA T4 và 16\,GB bộ nhớ RAM.
