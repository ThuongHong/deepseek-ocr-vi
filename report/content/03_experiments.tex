\section{Thực nghiệm}

\subsection{Phân chia dữ liệu}
Bộ dữ liệu UIT-HWDB-word đã được tác giả chia thành hai tập train/test riêng biệt. Ta sẽ không sử dụng tập validation mà thực hiện đánh giá trên tập test sau khi kết thúc quá trình huấn luyện.

Trong quá trình huấn luyện, dữ liệu sẽ được xử lý bởi lớp \texttt{DeepSeekOCRDataCollator}, sau đó được chuyển cho \texttt{Trainer} để thực hiện quá trình huấn luyện.

\subsection{Phương pháp thực nghiệm}
Ta dùng tập test để đánh giá mô hình trước và sau khi huấn luyện:

\begin{enumerate}
    \item Baseline (Zero-shot): Mô hình DeepSeek-OCR gốc (pre-trained) chưa qua bất kỳ quá trình huấn luyện thêm nào.
    \item Fine-tuned Model: Mô hình sau khi đã được huấn luyện với kỹ thuật LoRA trên tập dữ liệu UIT-HWDB-word.
\end{enumerate}

\subsection{Metrics}
Sau khi huấn luyện, mô hình được đánh giá trên tập kiểm thử dựa trên CER. CER (Character Error Rate) là đủ để đánh giá với dữ liệu ở cấp độ từ (word level). Nếu sử dụng WER (Word Error Rate) sẽ không phù hợp vì mỗi mẫu chỉ chứa một từ duy nhất.

Ngoài CER là metric chính, ta cũng xem xét thêm exact match rate (tỷ lệ dự đoán chính xác hoàn toàn), và accuracy (= 1 - CER) để có cái nhìn tổng quan hơn về hiệu suất mô hình.
