\section{Giới thiệu}

\subsection{Tổng quan về OCR và Thách thức với Tiếng Việt}
Nhận dạng ký tự quang học (OCR - Optical Character Recognition) là công nghệ chuyển đổi hình ảnh chứa văn bản (tài liệu in, viết tay, ảnh chụp) thành định dạng văn bản máy tính có thể chỉnh sửa và tìm kiếm được.

Tuy nhiên, việc áp dụng OCR cho tiếng Việt gặp nhiều thách thức do đặc thù ngôn ngữ và chữ viết. Một số thách thức chính bao gồm:

\begin{itemize}
    \item Đặc điểm ngôn ngữ: Tiếng Việt có nhiều dấu thanh và ký tự đặc biệt, điều này làm cho việc nhận diện trở nên khó khăn hơn so với các ngôn ngữ khác.
    \item Chất lượng hình ảnh: Các tài liệu quét có thể bị mờ, nghiêng hoặc có độ phân giải thấp, ảnh hưởng đến khả năng nhận diện của hệ thống OCR.
    \item Tính đa dạng của font chữ: Tiếng Việt sử dụng nhiều kiểu chữ khác nhau, từ chữ in đến chữ viết tay, điều này đòi hỏi hệ thống OCR phải được huấn luyện với một tập dữ liệu phong phú và đa dạng.
\end{itemize}

\subsection{Tổng quan về mô hình DeepSeek-OCR}
(\cite{wei2025deepseekocrcontextsopticalcompression}) DeepSeek-OCR là một mô hình Ngôn ngữ - Thị giác (Vision-Language Model - VLM) tiên tiến, được thiết kế theo hướng tiếp cận "nén ngữ cảnh quang học" (optical context compression). Thay vì chỉ nhận dạng ký tự đơn lẻ, mô hình xử lý toàn bộ hình ảnh tài liệu như một chuỗi token thị giác nén để giải mã ra văn bản.

\subsubsection{Kiến trúc}
Mô hình sử dụng kiến trúc End-to-End gồm hai thành phần chính nối tiếp nhau:

\paragraph{Encoder (DeepEncoder)} \hfill\\
Đây là bộ mã hóa thị giác tùy chỉnh với khoảng 380M tham số, được thiết kế để xử lý ảnh độ phân giải cao nhưng vẫn tối ưu hóa bộ nhớ. DeepEncoder bao gồm 3 module con:
\begin{itemize}
    \item Visual Perception: Sử dụng SAM-base (80M tham số) với cơ chế *window attention* để trích xuất các đặc trưng chi tiết cục bộ.
    \item Compressor: Một module tích chập (Conv layer) thực hiện giảm mẫu (downsample) 16 lần, giúp nén đáng kể số lượng vision tokens (ví dụ giảm từ 4096 xuống còn 256 tokens).
    \item Visual Knowledge: Sử dụng CLIP-large (300M tham số) với cơ chế *global attention* để nắm bắt ngữ nghĩa toàn cục và tri thức từ các token đã nén.
\end{itemize}

\paragraph{Decoder} \hfill\\
Về decoder, DeepSeek-OCR sử dụng mô hình ngôn ngữ lớn DeepSeek3B-MoE (Mixture-of-Experts). Mặc dù có tổng cộng 3B tham số, mô hình chỉ kích hoạt khoảng 570M tham số trong quá trình suy luận, đảm bảo tốc độ xử lý nhanh và hiệu quả cao

\subsubsection{Đầu vào}
DeepSeek-OCR có khả năng xử lý linh hoạt các loại đầu vào thông qua cơ chế Đa phân giải (Multi-resolution support):
\begin{itemize}
    \item Native Resolution: Hỗ trợ các chế độ phân giải cố định như Tiny, Small, Base và Large (tối đa 1280x1280 pixel).
    \item Dynamic Resolution (Gundam Mode): Hỗ trợ xử lý ảnh kích thước cực lớn hoặc tỷ lệ khung hình đặc biệt (như trang báo, tài liệu dài) bằng cách chia nhỏ ảnh (tiling) kết hợp với cái nhìn toàn cảnh, giúp không bị mất chi tiết.
\end{itemize}


\subsubsection{Đầu ra}
Mô hình hỗ trợ đa dạng định dạng đầu ra phục vụ nhiều mục đích khác nhau:
\begin{itemize}
    \item Văn bản thuần \& Markdown: Cho các tài liệu văn bản thông thường.
    \item HTML: Để nhận dạng và tái tạo cấu trúc bảng biểu (Tables) phức tạp.
    \item LaTeX / SMILES: Dành cho việc nhận dạng công thức toán học và công thức hóa học.
    \item Tọa độ (Bounding boxes): Cung cấp vị trí của đối tượng/văn bản cho các tác vụ grounding hoặc detection.
\end{itemize}